\section{Назначение узла, краткое описание его конструкции и анализ технических требований на сборку}

\subsection{Назначение узла}

Бак наддува применяется конструкции топливных систем ракет с жидкостными ракетными двигателями. Он необходим для поддержания давления наддува в топливных баках с целью предотвращения кавитации.

\subsection{Описание конструкции}

Конструкция бака наддува состоит их четырех деталей:
\begin{itemize}
    \item Полусфера нижняя
    \item Полусфера верхняя
    \item Фланец
    \item Штуцер
\end{itemize}

Толщина стенок полусфер составляет 6 мм. В ответственных местах имеется утолщение до 8 мм: У верхней полусферы --- вблизи места соединения с фланцем, также у обеих полусфер --- в месте их соединения. Полусферы имеют центровочную поверхность $\varnothing 526 \; \t{мм}$ для облегчения операции сборки. Фланец предназначен для соединения штуцера с верхней полусферой. Особенностью фланца являются его конические поверхности. Они обеспечивают легкость выполнения сварочных операций. Штуцер предназначен для соединения узла с остальными элементами топливной системы ракеты.

\subsection{Технические требования, предъявляемые у узлу}

Перечислим основные характеристики узла:
\begin{itemize}
    \item рабочий продукт - гелий
    \item номинальный объем - $0.075 \; \t{м}^3$
    \item материал - титановый сплав ВТ14
    \item герметичность - $1.0 \cdot 10^{-6} \; \t{Вт}$
    \item рабочее давление - $25 \; \t{МПа}$
    \item срок эксплуатации - не менее 10 лет
\end{itemize}

Исходя из условий эксплуатации, к узлу предъявляются следующие требования:
\begin{enumerate}
    \item Сварные швы по ОСТ 26-1-87.
    
    Данный отраслевой стандарт распространяется на сварные соединения в конструкциях из титана и титановых сплавов и устанавливает основные типы и конструктивные элементы сварных швов.
    \item Категория сварного шва - I.
    
    Поскольку сварной шов применяется в ответственном узле, хранящем гелий под высоким давлением, к нему предъявляются высокие требования качества.
    \item Сварка электронно-лучевая по ГОСТ 3044-79.
    
    Электронно-лучевая сварка обеспечивает высокий уровень чистоты сварочной ванны. Титан очень реактивен при высоких температурах и может легко взаимодействовать с кислородом, азотом и другими газами, что приводит к ухудшению механических свойств. Поскольку электронно-лучевая сварка проводится в вакууме, это предотвращает контакт титана с атмосферными газами, обеспечивая высокое качество сварного шва.
    
    Также электронно-лучевая сварка обеспечивает глубокое проплавление при минимальной зоне температурного влияния. Это позволяет сваривать детали из титана с минимальными деформациями и внутренними напряжениями, а также уменьшает вероятность непровара.

    Кроме того возможность точной фокусировки электронного пучка позволяет выполнять сварку с высокой точностью, что особенно важно для сложных и ответственных конструкций из титана.
    \item Штамповка по ОСТ 92-1675-87.
    
    Данный отраслевой стандарт распространяется на штамповку листовых деталей и заготовок из титановых сплавов. Он устанавливает требования к технологическим операциям изготовления листовых деталей и заготовок и схемы типовых технологических процессов.
    \item Нормы прочности по ГОСТ Р 56514-2015.
    
    Данный стандарт устанавливает нормы прочности для всех этапов эксплуатации автоматических одноразовых аппаратов (АКА), а также требования к определению нагрузок, расчетной проверке прочности, экспериментальной отработке прочности, контролю и подтверждению прочности на этапах экспериментальной отработки, летных испытаний и эксплуатации АКА. Испытания на прочность являются обязательным для ответственных изделий в ракетно-космической техники.
    \item Правила проведения пневмоиспытаний по РД 26-12-29-88, испытания на герметичность по ГОСТ 28517-90.
    
    РД 26-12-29-88 <<Правила проведения пневматических испытаний изделий на прочность и герметичность>> устанавливает организацию и порядок проведения работ и общие требования безопасности при проведении пневматических испытаний, а также к устройству, размещению и эксплуатации стендов, установок и сооружений, предназначенных для этих целей.

    ГОСТ 28517-90 <<Масс-спектрометрический метод течеискания>> устанавливает общие требования к контролю герметичности. Его применяют для проведении испытаний на герметичность при регистрации потоков в диапазоне от $10^{-14}$ до $10^{-2} \; \t{Вт}$.
    \item Рентгеноконтроль сварных швов по ГОСТ 7512-82.
    
    Радиографический контроль применяют для выявления в сварных соединениях трещин, непроваров, пор, шлаковых, вольфрамовых, окисных и других включений. Его также применяют для выявления прожогов, подрезов, оценки величины выпуклости и вогнутости корня шва, недопустимых для внешнего осмотра.
\end{enumerate}
