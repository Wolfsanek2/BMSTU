\documentclass[a4paper, 12pt]{article}

\usepackage{extsizes} % Вроде как для 14 шрифта
\usepackage{mathtext}

\usepackage{geometry} % отступы
\geometry{top=2cm}
\geometry{bottom=2cm}
\geometry{left=2.5cm}
\geometry{right=1.5cm}

\linespread{1.33} % 1.33 - полуторный межстрочный интервал

\usepackage{polyglossia} % для работы с языками
\setdefaultlanguage{russian}

\setmainfont[Ligatures={TeX}]{Times New Roman} % шрифт по умолчанию

\usepackage{amsmath, amsfonts, amssymb, amsthm, mathtools} %пакеты от американского математического общества

\RequirePackage{amsmath,amssymb,latexsym}

\usepackage{graphicx} % для картинок
\graphicspath{{./Рисунки/}} % Папка с картинками

\usepackage[tableposition=top]{caption} % Для подписей рисунков и таблиц
\DeclareCaptionLabelSeparator{defffis}{ ~---~ }
\DeclareCaptionLabelFormat{gostfigure}{Рисунок #2}
\DeclareCaptionLabelFormat{gosttable}{Таблица #2}
\captionsetup[figure]{justification=centering, labelsep=defffis, format=plain}
\captionsetup[table]{justification=raggedright, labelsep=defffis, format=plain, singlelinecheck=false}
\captionsetup[figure]{labelformat = gostfigure}
\captionsetup[table]{labelformat = gosttable}



\usepackage{setspace} % Изменение межстрочного интервала
\usepackage{float} % Для использования [H] при вставке картинок
\usepackage{tabularx} % Таблицы с автоматической шириной столбцов
\usepackage{multirow} % Для объединения строк в таблице
\usepackage{gensymb} % Для математических символов, включая градус
\usepackage{textcomp} % Для градусов

\usepackage{hyperref} % Для гиперссылок
\hypersetup{
    colorlinks=true,
    linkcolor=blue,
}

% \usepackage{ulem} % Для подчеркиваний

\AtBeginDocument{\numberwithin{equation}{section}}
\AtBeginDocument{\numberwithin{table}{section}}
\AtBeginDocument{\numberwithin{figure}{section}}

\usepackage{sectsty} % Для настройки шрифта заголовков
\allsectionsfont{\singlespacing} % Одинарный интервал в заголовках

\setlength{\parindent}{5ex} % Абзацный отступ 5 символов
\usepackage{indentfirst} % Для абзацного отступа в первой абзаце после заголовка

% Настройка заголовков
\usepackage{titlesec}
\titleformat{\section}{\normalfont \Large}{\thesection}{1em}{}
\titleformat{\subsection}{\normalsize \large}{\thesubsection}{1em}{}
\titleformat{\subsubsection}{\normalsize}{\thesubsubsection}{1em}{}
% Настройка горизонтальных отступов заголовков
\titlespacing*{\section}{\parindent}{*4}{*4}
\titlespacing*{\subsection}{\parindent}{*4}{*4}
\titlespacing*{\subsubsection}{\parindent}{*4}{*4}

%Это чтобы сделать красивое оглавление
\makeatletter
\renewcommand{\l@section}{\@dottedtocline{1}{0em}{1.25em}}
\renewcommand{\l@subsection}{\@dottedtocline{2}{1.25em}{1.75em}}
\renewcommand{\l@subsubsection}{\@dottedtocline{3}{2.75em}{2.6em}}
\makeatother

\setlength{\belowcaptionskip}{-16pt} % Уменьшение расстояние между подписью и русунком

\renewcommand{\t}{\text}
\renewcommand{\phi}{\varphi}
\renewcommand{\epsilon}{\varepsilon}

\def \tabfrac#1#2{\displaystyle \frac{\strut #1}{\strut #2}}
\def \matrfrac#1#2{\displaystyle \frac{#1}{#2}}